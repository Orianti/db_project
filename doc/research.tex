\chapter{Исследовательский раздел}

\section{Технические характеристики тестирующего ПК}

Тестирование приложения производилось на локальном сервере, запущенном на персональном компьютере со следующими техническими характеристиками:
\begin{itemize}
	\item Процессор Intel Core i5-6200U с тактовой частотой 2.3 ГГц.
	\item Оперативная память объемом 6 Гб.
	\item SSD со скоростью чтения и записи 540 МБ/с.
	\item Операционная система Ubuntu 20.04.
\end{itemize}

\section{Исследование времени добавления записей в базу данных}

Для исследования производительность базы данных использовался модуль Faker, позволяющий генерировать случайные данные различных типов.~\cite{faker}

С помощью модуля Faker было подготовлено в общей сложности 10000 записей для добавления в различные таблицы базы данных. Данные загружались группами, начиная с 1000 записей, с шагом в 1000 записей. В результате исследования получен график, представленный на рисунке \ref{plt:res}. Полученные данные о времени добавления записей в базу данных были усреднены.

Время добавления в базу данных 10000 записей можно считать приемлемым. Можно предположить, что на практике именно такой объем данных и будет единовременно загружаться в систему.

\begin{figure}
\centering
\begin{tikzpicture}
	\begin{axis}[
		width=0.85\textwidth,
		ylabel={Время, с}, 
		xlabel={Количество записей},
	]
		\addplot coordinates {
			(1000,0.4) (2000,1.6) (3000,2.7) (4000,4.3) (5000,6.8) (6000,8.2) (7000,10.5) (8000,12) (9000,13) (10000,15.8)
		};
	\end{axis}
	
\end{tikzpicture}
\caption{Результаты исследования}
\label{plt:res}
\end{figure}

\section*{Выводы}

В экспериментальном разделе расчетно-пояснительной записки было исследовано время загрузки данных в систему.