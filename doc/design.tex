\chapter{Конструкторский раздел}

\section{Проектирование базы данных}

В аналитическом разделе расчетно-пояснительной записки на основе данных о предметной области была построена ER-диаграмма. Для проектирования базы данных преобразуем диаграмму <<сущность --- связь>> в схему базы данных на основе реляционной модели данных.

Выделим следующие таблицы:
\begin{itemize}
	\item Камера (Camera) --- основная таблица базы данных, содержит внешние ключи к таблицам Местоположение, Спецификации, Состояние.
	\item Местоположение (Location) --- таблица местоположений дорожных камер, содержит внешний ключ к таблице Город.
	\item Город (City) --- таблица городов.
	\item Спецификации (Specification) --- таблица технических данных о камерах, содержит внешний ключ к таблице Производитель.
	\item Производитель (Producer) --- таблица производителей камер, содержит внешний ключ к таблице Город.
	\item Состояние (State) --- таблица состояний камер.
	\item Сервис (Service) --- таблица, хранящяя данные о сервисных обслуживаниях камер, содержит внешние ключи к таблицам Камера и Сервисная организация.
	\item Сервисная организация (ServiceOrganization) --- таблица организаций, предоставляющих сервисное обслуживание камер, содержит внешний ключ к таблице Город.
\end{itemize}

На рисунке \ref{img:dbd} представлена диаграмма базы данных.

\imgwc{h}{0.95\textwidth}{dbd}{Диаграмма базы данных}

\section{Архитектура приложения}

Для проектирования приложения применим архитектурый шаблон проектирования Model-View-Controller (далее --- MVC).~\cite{mvc}

\subsection{Архитектурный шаблон MVC}

MVC представляет из себя схему разделения данных и бизнес-логики приложения, пользовательского интерфейса и управляющей логики на три независимых компонента: модель, представление и контроллер. Такой подход позволяет изолировать данные и управляющую логику, независимо разрабатывать, тестировать, поддерживать и модифицировать компоненты. Схема шаблона MVC представлена на рисунке \ref{img:mvc}.

\imgwc{h}{0.7\textwidth}{mvc}{Схема шаблона проектирования MVC}

\subsubsection{Модель}

Модель представляет собой данные и методы для работы с данными. В модели выполняются запросы к базе данных, бизнес-логика. Этот компонент разрабатывается таким образом, чтобы отвечать на запросы контроллера, изменять свое внутреннее состояние и не зависеть от представлений.

\subsubsection{Представление}

Представление получает данные модели и отображает их пользователю. Представление не обрабатывает данные.

\subsubsection{Контроллер}

Контроллер является связующим компонентом --- интерпретирует действия пользователя, оповещая модель об изменениях, которые необходимо внести.

\subsection{UML-диаграммы классов}

На рисунках \ref{img:uml_io}--\ref{img:uml_view} представлены UML-диаграммы классов приложения.

\imgwc{h}{0.95\textwidth}{uml_io}{UML-диаграмма классов \code{Model} и \code{Manager}}
\imgwc{h}{0.5\textwidth}{uml_view}{UML-диаграмма классов \code{View} и \code{FormView}}

\subsection{UML-диаграмма компонентов приложения}

На рисунке \ref{img:appd} представлена UML-диаграмма модулей приложения.

\imgwc{h}{0.95\textwidth}{appd}{UML-диаграмма компонентов приложения}

\section*{Выводы}

В конструкторском разделе была спроектирована база данных и архитектура приложения, построены ULM-диаграммы классов приложения и UML-диаграмма компонентов приложения.