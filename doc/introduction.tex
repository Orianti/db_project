\chapter*{Введение}
\addcontentsline{toc}{chapter}{Введение}

В настоящее время сеть автомобильных дорог в России стремительно расширяется: увеличивается как протяженность, так и пропускная способность, открываются всё новые и новые трассы~\cite{road}. Одним из основных инструментов контроля автомобильного потока являются камеры безопасности дорожного движения~\cite{camera}. Появившиеся еще в 1960-x годах, к XXI веку дорожные камеры получили широкое распространение и стали цифровыми. Современные дорожные камеры могут быть оснащены сетевым соединением, что позволяет ускорить процесс обработки данных, технического обслуживания и мониторинга.

С развитием компьютерной техники появилась возможность автоматизации множества процессов. Так, например, автоматизация процессов мониторинга и технического обслуживания камер безопасности дорожного движения позволит упростить процессы наблюдения за работоспособностью камер и сервисного обслуживания.

\textbf{Целью} данной курсовой работы являлась разработка приложения, обеспечивающего частичную автоматизацию процессов мониторинга и технического обслуживания камер безопасности дорожного движения.

Достижению цели способствовали следующие поставленные \textbf{задачи}:
\begin{itemize}
	\item анализ предметной области, требований к базе данных и приложению в целом;
	\item проектирование базы данных и приложения;
	\item реализация приложения;
	\item тестирование приложения, исследование его стабильности и быстродействия.
\end{itemize}