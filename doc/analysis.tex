\chapter{Аналитический раздел}

\section{Описание предметной области}

Использовать приложение для контроля работы дорожных камер будут сотрудники ГИБДД, а, следовательно, оно должно не требовать от пользователя специальных навыков, быть простым в установке и эксплуатации. 

Выделим наиболее важные свойства и характеристики дорожных камер:
\begin{itemize}
	\item местоположение;
	\item спецификации (например, тип камеры, производитель и пр.);
	\item текущее состояние камеры (например, исправное или неисправное).
\end{itemize}

Кроме того, для каждой камеры могут быть определены данные о сервисном обслуживании: название организации проводящей обслуживание, причины обслуживания, результаты обслуживания и пр.

\section{Требования к приложению}

Проанализировав предметную область курсовой работы, определим требования к приложению.

В качестве общей архитектурной концепции следует выбрать Web-приложение. Оно не привязывает пользователя к конкретной операционной системе, не требует установки и работает в привычной для обычного пользователя среде --- браузере.

Для хранения информации о камерах следует использовать базу данных. В приложении должен быть предусмотрен простой механизм для перехода на другую систему управления базами данных (далее --- СУБД).

Так как хранящиеся данные могут отражать различные аспекты мониторинга и технического обслуживания, следует разделить пользователей приложения на группы с различными правами доступа. Каждая группа пользователей должна получать доступ к определенным разделам приложения. Регистрировать новых пользователей и предоставлять права доступа должен администратор приложения. Выделим группы пользователей:
\begin{itemize}
	\item администраторы;
	\item аудиторы (занимаются мониторингом дорожных камер и оформлением заявок на сервисное обслуживание);
	\item юристы (занимаются проверкой документов, например договоров).
\end{itemize}
В приложении должен быть предусмотрен простой механизм регистрации новых пользователей и создания новых групп пользователей.

\section{Логическая модель данных}

Исходя из описания предметной области и требований к приложению можно сделать вывод, что следует выбрать реляционную модель данных.

\subsection{Диаграмма <<сущность --- связь>>}

На рисунке \ref{img:erd} представлена диаграмма <<сущность --- связь>>.

\imgwc{h}{0.95\textwidth}{erd}{ER-диаграмма}

\section{Обзор существующих СУБД}

Рассмотрим существующие реляционные системы управления базами данных.

\subsection{MS SQL}

Microsoft SQL Server разрабатывается корпорацией Microsoft с 1989 года. Основным языком запросов в этой СУБД является Transact-SQL --- реализация стандарта ANSI/ISO по языку SQL с расширениями. Работает в операционных системах UNIX, OS/2, Windows. Является одной из наиболее популярных СУБД и отличается высокой надежностью, производительностью и большим сообществом разработчиков.~\cite{mssql}

Преимущества и недостатки:
\begin{itemize}
	\item[$+$] Легко маштабируемая СУБД.
	\item[$+$] Автоматизированные административные задачи: управление блокировкой, управление память и пр.
	\item[$+$] Поддержка большинства инструкций языка SQL.
	\item[$+$] Простая интеграция с другими продуктами Microsoft.
	\item[$-$] Зависимость от операционной среды.
	\item[$-$] Высокая цена.
\end{itemize}

\subsection{MySQL}

MySQL --- свободная реляционная СУБД, разрабатываемая Oracle. Программный продукт распространяется по лицензии GNU General Public License и коммерческой лицензии. MySQL обеспечивает поддержку большого количества типов таблиц.~\cite{mysql}

Преимущества и недостатки:
\begin{itemize}
	\item[$+$] Легко маштабируемая СУБД.
	\item[$+$] Высокая производительность за счет упрощения некоторых стандартов.
	\item[$+$] Поддержка большинства инструкций языка SQL.
	\item[$+$] Существуют версии, распространяемые по лицензии GNU.
	\item[$-$] Заложены некоторые ограничения функционала, которые иногда необходимы в особо требовательных приложениях.
	\item[$-$] Уступает другим СУБД по надежности.
\end{itemize}

\subsection{PostgreSQL}

Свободная система управления базами данных PostgreSQL существует для множества UNIX-подобных операционных систем и Microsoft Windows. Часто используется при разработке веб-сайтов. В отличии от других СУБД, поддерживает некоторые важные объектно-ориентированные и реляционные функции баз данных.~\cite{postgresql}

Преимущества и недостатки:
\begin{itemize}
	\item[$+$] Легко маштабируемая СУБД.
	\item[$+$] Поддержка множества сторонних библиотек, инструментов для проектирования и управления данными.
	\item[$+$] Поддержка большинства инструкций языка SQL.
	\item[$+$] Открытый исходный код.
	\item[$-$] Производительность в некоторых задачах ниже, чем у MySQL.
\end{itemize}

\subsection{SQLite}

SQLite --- это встраиваемая СУБД (библиотека). Используется преимущественно в небольших проектах и приложениях, по умолчанию встроена в мобильные приложения на платформе Android. База данных использует для своей работы единственный файл и не предоставляет сетевого интерфейса.~\cite{sqlite}

Преимущества и недостатки:
\begin{itemize}
	\item[$+$] Повышение производительности за счет хранения базы данных на стороне клиента.
	\item[$+$] Хорошо переносимая, так как представляет из себя единственный файл.
	\item[$+$] Хорошо подходит для небольших приложений.
	\item[$-$] Накладывает множество ограничений на разработчика.
\end{itemize}

\section*{Выводы}

В аналитическом разделе определена задача и предметная область курсовой работы. Исходя из анализа предметной области определены требования к приложению, в том числе требования, направленные на упрощение процесса модификации. Определена логическая модель базы данных, построена диаграмма <<сущность --- связь>>. Рассмотрены наиболее популярные системы управления базами данных.
