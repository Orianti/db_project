\chapter{Технологический раздел}

\section{Выбор языка программирования, среды и инструментов разработки}

В качестве языка программирования был выбран язык Python. Данный язык программирования активно используется в веб-разработке, имеет множество сторонних библиотек и фреймворков, что облагчает разработку.~\cite{python}

В качестве фреймворка для разработки был выбран Django~\cite{django}, так как он содержит мощные инструменты для разработки web-приложений и является одним из самых популярных фреймворков. Для верстки шаблонов сайта был выбран фреймворк Bootstrap~\cite{bootstrap}.

В качестве интегрированной среды разработки был выбран PyCharm. Эта среда поддерживает Django, содержит встроенный статический анализатор кода, множество встроенных инструментов упрощающих разработку и отдадку приложения.~\cite{pycharm}

В качестве СУБД будет использована PostgreSQL (SQLite в процессе разработки и отладки приложения).~\cite{postgresql}~\cite{sqlite}

\section{Реализация архитектуры программы}

\subsection{Реализация шаблонов проектирования}

В основе Django лежит шаблон проектирования MVC, который во фреймворке называется Model-View-Template, где Model --- модель, являющаяся фактически ORM-сущностью, View --- контроллер, Template --- представление. Бизнес-логику в Django принято выделять в отдельный компонент.

\subsection{Компоненты приложения}

Выделим четыре компонента: компонент доступа к данным, компонент бизнес-логики, компонент графического интерфейса пользователя, компонент, связывающий бизнес-логику и графический интерфейс пользователя.

Полные листинги с исходным кодом основных классов представлены в приложениях.

\subsubsection{Компонент доступа к данным}

Данный компонент представляет из себя классы \code{django.db.models.Models} (далее --- модели), данные в которых соответствуют атрибутам таблиц в базе данных. Модели содержат методы для обработки данных на уровне строки, например метод \code{get\_state()}, возвращающий словесное описание состояния камеры. На листинге \ref{lst:model} представлена модель \code{State}, соответствующая таблице Состояние (State) в базе данных. Класс также содержит метаинформацию для упрощения работы в приложении.

\lstinputlisting[language=python, caption={Модель \code{State}}\label{lst:model}]
{inc/lst/model.py}

Для работы с данными на уровне таблицы используются классы, называемые менеджерами. Эти классы наследуются от \code{django.db.models.Managers} и содержат методы для доступа к данным. На листинге \ref{lst:manager} приведен пример менеджера.

\lstinputlisting[language=python, caption={Менеджер \code{CameraManager}}\label{lst:manager}]
{inc/lst/manager.py}

\subsubsection{Компонент бизнес-логики}

Компонент бизнес-логики состоит из классов, которые было решено называть сервисами (services). На листинге \ref{lst:service} представлен класс, контролирующий работу камер. Метод этого класса возвращает список камер, состояние которых неисправно и для которых, вместе с тем, еще не было офорлено сервисное обслуживание.

\lstinputlisting[language=python, caption={Сервис \code{NotServicedCamerasController}}\label{lst:service}]
{inc/lst/service.py}

\subsubsection{Компонент, связывающий бизнес-логику и интерфейс}

Классы-контроллеры в Django наследуются от класса \code{django.views.generic.DefaultView}. При этом существуют классы для выполнения типичных задач представления данных: для отображения списка объектов --- \code{ListView}, для отображения информации о конкретном объекте --- \code{DetailView} и др. На листинге \ref{lst:view} представлен класс \code{AuditDetailView}. В нем, в частности, переопределен метод \code{get\_context\_data()} для передачи в компонент интерфейса дополнительной информации об обслуживании камеры.

\lstinputlisting[language=python, caption={Контроллер \code{AuditDetailView}}\label{lst:view}]
{inc/lst/view.py}

\subsubsection{Компонент интерфейса}

Компонент интерфейса представляет из себя набор \code{HTML}-страниц, использующих шаблонизатор Django, который позволяет использовать шаблоны для генерации конечных страниц. Шаблонизатор позволяет, в частности, переиспользовать код и ускоряет верстку Web-приложения. Данные передаются из контроллера в качестве параметров на страницу.

\section{Интерфейс приложения}

Интерфейс приложения представляет из себя страницу, в шапке которой расположено название приложения, информация об авторизированном пользователе и кнопка выхода из системы.

В основной части приложения находится навигационное меню, в котором расположены ссылки на доступные пользователю страницы приложения. Ниже расположено название текущей страницы и непосредственно контект страницы, например таблицы с данными. Имеются также кнопки, например для добавления новых записей.

В нижней части страницы находится служебная информация и ссылка на административную панель сайта.

На рисунках \ref{img:interface}--\ref{img:admin} представлен интерфейс приложения.

\imgwc{h}{0.95\textwidth}{interface}{Интерфейс страницы <<Аудит камер>>}
\imgwc{h}{0.95\textwidth}{admin}{Интерфейс административной панели}

\section{Отладка и тестирование приложения}

Для процесса отладки программы использовался отладчик и статический анализатор кода, встроенные в среду разработки PyCharm.

Программа успешно прошла все тесты и показала высокую стабильность.

\section*{Выводы}

В технологическом разделе выбран язык и инструменты разработки, реализовано Web-приложение и проведено его успешное тестирование.